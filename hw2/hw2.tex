\documentclass[a4paper, 11pt]{exam}
\usepackage{titling}
\newcommand{\subtitle}[1]{%
  \posttitle{%
    \par\end{center}
    \begin{center}\large#1\end{center}
    }%
}

\usepackage{url}
\usepackage{amsmath,amsthm,enumitem,amssymb}
\usepackage{graphicx}
\usepackage{hyperref}
\usepackage{float}
\usepackage{listings}
\renewcommand{\labelenumii}{\roman{enumii}}

\title{Homework Assignment 5}
\subtitle{CS 6460\\
Apr 20,2018\\
Marek Baranowski}

\begin{document}
\maketitle
\begin{enumerate}
\item Chapter 5 of OSC discusses possible race conditions on various
  kernel data structures. Most scheduling algorithms maintain a run
  queue, which lists processes eligible to run on a processor. On
  multicore systems, there are two general options: (1) each
  processing core has its own run queue, or (2) a single run queue is
  shared by all processing cores. Briefly give at least one advantage
  and one disadvantage of each of these approaches.

Run queues on each core:

Pros:
\begin{enumerate}
\item Run queues on each core is helpful because the scheduling
  decisions are distributed, and the core can respond to interrupts
  better.
\item Core affinity is baked into this design, leading to better cache
  peformance.
\end{enumerate}
Cons:
\begin{enumerate}
\item Distributing the run queues is more difficult if the scheduler
  wants to avoid idle or overloaded cores.
\item The run queues have to be carefully managed to avoid losing a
  process or accidentally running one process on two cores.
\end{enumerate}

A global run queue:
Pros:
\begin{enumerate}
\item The scheduler design is simpler, making bugs less likely.
\item It is to design the scheduler to better utilize all resources.
\end{enumerate}

Cons:
\begin{enumerate}
\item The scheduler design would be more complicated to keep process
affinity.
\item The scheduler is run in an inherently serial way, potentially
wasting resources when making a scheduling decision.
\item The global scheduler is always communicating with other cores.
\end{enumerate}

This article talks about several bugs in the distributed Linux CFS scheduler
where significant waste of resources was found.
\begin{verbatim}https://blog.acolyer.org/2016/04/26/the-linux-scheduler-a-decade-of-wasted-cores/\end{verbatim}


\item Consider the following code example for allocating and releasing processes:

\begin{lstlisting}[numbers=left]
#define MAX_PROCESSES 255
int number_of_processes = 0;

/* the implementation of fork() calls this function */
int allocate_process() {
  int new_pid;
  if (number_of_processes == MAX_PROCESSES) {
    return -1;
  } else {
    /* allocate necessary process resources */
    ++number_of_processes;
    return new_pid;
  }
}

/* the implementation of exit() call this function */
void release_process() {
  /* release process resources */
  --number_of_processes;
}
\end{lstlisting}

\begin{enumerate}
\item Identify the race condition(s).
\begin{enumerate}
\item There are two read/write races between line 7 and lines 11 and 19.
\item Lines 12 and 19 have read/write and write/write races.
\item Line 13 races with itself if two threads are at this line of
  code. Similarly for 19.
\end{enumerate}
\item There needs to be an acquire before lines 7 and 19. There needs to be a
release before lines 8 and 11 and after line 19. This eliminates
all previously mentioned data races. 
The increment needs to be atomic with the comparison, otherwise the the number
of processes could change and we could accidentally exceed MAX\_PROCESSES. This
is why the lock in allocate\_process is not fine grained.

\item Without rewriting the program, we cannot simply change the data
  types to atomic. This would make the program race free, however we
  could violate the invariant that number\_of\_processes $<$
  MAX\_PROCESSES, as previously mentioned. The check needs to be
  atomic with the increment.
\end{enumerate}

\item Briefly explain why the OS often uses an FCFS disk-scheduling
  algorithm when the underlying device is an SSD.

\item Describe one advantage and two disadvantages of using SSDs as a
  caching tier compared with using only magnetic disks.

\begin{itemize}
\item An advantage is the very high bandwidth, low latency nature of
  SSD media. This makes page swapping less costly.
\item A disadvantage is that swapping tends to read and write
  different data to the disk, increasing wear on the disk.
\item Another disadvantage is if the disk is nearly full, the erase
  block property of the SSD may lead to write amplification, and
  ultimately lower access times than a regular magnetic disk.
\end{itemize}
\item Consider a file system that uses inodes to represent files. Disk
  blocks are 8 KB in size, and a pointer to a disk block requires 4
  bytes. This filesystem has 12 direct disk blocks, as well as one
  single, one double, and one triple indirect disk blocks entry in its
  inode. What is the maximum size of a file that can be stored in this
  file system?




\end{enumerate}


\end{document}